\def\CC{{C\nolinebreak[4]\hspace{-.05em}\raisebox{.4ex}{\tiny\bf ++}}}
\usepackage{multirow}

\author{Louis Onrust}
\title{$p(\text{conclusions} | \text{Skipping \{*2*\}})$}
\subtitle{Bayesian Language Modelling with Skipgrams}
\date{}
\begin{document}

\begin{frame}
    \titlepage
\end{frame}
\note[itemize]{
}

\begin{frame}{Bayesian Language Modelling with Skipgrams}
    \begin{block}{}
        Louis Onrust \\
        Centre for Language Studies, Radboud University \\
        Center for Processing Speech and Images, KU Leuven
    \end{block}

    \begin{block}{}
        \href{mailto:l.onrust@let.ru.nl}{l.onrust@let.ru.nl} \\
        \href{https://github.com/naiaden}{github.com/naiaden}
    \end{block}
\end{frame}
\note[itemize]{
}

\begin{frame}{Scope of the Project}
    \begin{block}{Scope}
        \begin{itemize}
            \item Language models
            \item Latent variable models
            \item Domain-dependence of LVLM
            \item Intrinsic \& extrinsic evaluation
        \end{itemize}
    \end{block}

    \begin{block}{Goal}
        \begin{itemize}
            \item Bring back language modelling in Bayesian language modelling
            \item Improve cross domain langauge modelling with skipgrams
        \end{itemize}
    \end{block}
\end{frame}
\note[itemize]{
}

\begin{frame}{Language Model}
    \begin{block}{Traditional method}
        The process:
        \begin{itemize}
            \item Read $n$-gram $p$
            \item Increment frequency of $p$
            \item Repeat, preferably ad infinitum
        \end{itemize}

        $n$-gram probabilities are then determined by their MLE
    \end{block}

    \begin{block}{Smoothed Traditional Language Model}
        What to do when the occurrence count of $p$ is 0?
        \begin{itemize}
            \item Not assign 0 as probability $\rightarrow$ smoothing
            \item Fall back to the last $(n-1)$ words of $p$ $\rightarrow$ backoff
        \end{itemize}

        One of the best methods is still Modified Kneser-Ney: backoff and smoothing
    \end{block}
\end{frame}
\note[itemize]{
}

\begin{frame}{Language Model}
    \begin{block}{Bayesian method}
        \begin{itemize}
            \item Assume texts are generated by some process
            \item Consider the texts to be a sample from the process
            \item Infer underlying process
        \end{itemize}
    \end{block}

    \begin{block}{Bayesian Unigram Language Model: Chinese Restaurant Process}
        \begin{itemize}
            \item Clusters are tables, unigram tokens are customers
            \item Initially tokens seat at the same type table
            \item In the inference step, customers get to choose a new identity
        \end{itemize}
    \end{block}

    \begin{block}{Bayesian $n$-gram Language Model: Nested Chinese Restaurant Process}
        \begin{itemize}
            \item Each context is a restaurant
            \item Each $n$ is a floor
            \item Each $n$-gram is a table
            \item Each $(n-1)$-gram sits at a table on the $(n-1)$th floor
        \end{itemize}
    \end{block}
\end{frame}
\note[itemize]{
}

\begin{frame}{Bayesian Language Model: Learning \& Estimating}
    \begin{block}{Chinese Restaurant Process: Empirical Distribution}
        \begin{itemize}
            \item Each $n$-gram enters the restaurant, and goes to the $n$th floor, to the room that represents the context
            \item There he seeks for the table with other $n$-grams of the same type
                \begin{itemize}
                    \item If there is such a table, he joins that table
                    \item Otherwise he seats himself at an empty table
                \end{itemize}
            \item For each new table, a family member of the same $n$-gram but of length $(n-1)$ is sent to represent the family
                \begin{itemize}
                    \item This process repeats for $0 < x \leq n$
                \end{itemize}
        \end{itemize}
    \end{block}

    \begin{block}{Chinese Restaurant Process: Inference}
        With $m$ customers in the restaurant, a customer re-enters the restaurant and sits a table $t$ with probability
        \begin{itemize}
            \item $\frac{1}{m+1}$ with another $n$-gram $p$, or $\frac{|t|}{m+1}$ at the same table as $p$
            \item $\frac{1}{n}$ at a new table
        \end{itemize}
        The number of tables grows logarithmically
    \end{block}
\end{frame}

\begin{frame}{Processes and Priors}
    \begin{block}{\only<1>{Reeeeeeeee-cap}\only<2>{REEMIX}}
        \begin{itemize}
            \item We described a Chinese restaurants process mixture model
                \begin{align}
                    \pi_{[M]} &\sim \operatorname{CRP}(M)\label{eq:CRPMMpartition} \\
                    \phi_t | \pi_{[M]} &\overset{\text{iid}}{\sim} G_0 && \text{ for }t\in\pi_{[M]},\label{eq:CRPMMlatent} \\
                    x_i|\phi,\pi_{[M]} &\overset{\text{ind}}{\sim} F(\phi_t) && \text{ for }t\in\pi_{[M]}\text{ and }i\in t,\label{eq:CRPMMdatapoints}
                \end{align}
        \end{itemize}
    \end{block}

    \only<1>{
        \begin{block}{Nested Pitman-Yor Chinese Restaurant Process}
            \begin{itemize}
                \item CRP and DPCRP give logarithmic growth
                \item Language manifests typically in power law growth
                \item PYCRP as generalisation of CRP and DPCRP
                    \begin{description}
                        \item[CRP] No parameters
                        \item[DPCRP] Concentration parameter $\alpha$
                        \item[PYCRP] Concentration parameter $\alpha$ and discount parameter $\gamma$
                    \end{description}
            \end{itemize}
        \end{block}
    }
    \only<2>{
        \begin{block}{Nested Pitman-Yor Chinese Restaurant Process Mixture Model}
            \begin{align}
                \pi_{[M]} &\sim \operatorname{PYCRP}(\alpha,\gamma,G_0) \\
                G_u &\sim\operatorname{PYCRP}(\alpha_{|u|},\gamma_{|u|},G_{\pi(u)}) \\
                \phi_t | \pi_{[M]} &\overset{\text{iid}}{\sim} G_0 && \text{ for }t\in\pi_{[M]}, \\
                x_i|\phi,\pi_{[M]} &\overset{\text{ind}}{\sim} F(\phi_t) && \text{ for }t\in\pi_{[M]}\text{ and }i\in t,
            \end{align}
        \end{block}
    }
\end{frame}
\note[itemize]{
}

\begin{frame}{Bayesian Language Model: The Implementation}
    \begin{block}{Motivation}
        Existing Bayesian language models\ldots
        \begin{itemize}
            \item are merely an algorithmic showcase without real language modelling aspirations
            \item cannot handle really big data sets
        \end{itemize}
    \end{block}

    \begin{block}{Implementation}
        We use the following software:
        \begin{description}
            \item[cpyp] an existing \CC{} framework on BNP with PYP priors
            \item[colibri] an existing \CC{} framework for pattern modelling
        \end{description}
    \end{block}

    \begin{block}{Advantages}
        \begin{itemize}
            \item We can now handle many patterns such as $n$-grams, skipgrams, and flexgrams
            \item Tresholding patterns on many levels
            %\item Efficient storage of patterns
        \end{itemize}
    \end{block}
\end{frame}
\note[itemize]{
}

\begin{frame}{Results: The Setup}
    \begin{block}{Data Sets}
        \begin{itemize}
            \item JRC English
            \item Google 1 billion words
            \item EMEA English
        \end{itemize}
    \end{block}

    \begin{block}{Backoff Methods}
        \begin{description}
            \item[$n$-gram] full recursive backoff to shorter $n$-grams
            \item[Limited] recursive backoff to all patterns $\leq n$ until match
            \item[Full] recursive backoff to all patterns $\leq n$
        \end{description}
    \end{block}

    \begin{block}{Evaluation Measure}
        \begin{itemize}
            \item Intrinsic evaluation with perplexity
        \end{itemize}
    \end{block}
\end{frame}
\note[itemize]{
}

\begin{frame}{Results: An Overview}
    \begin{block}{Summary}
        \begin{itemize}
            \item Within domain evaluation yields best performance
            \item Adding skipgrams increases performance on cross domain evaluation
            \item For generic corpora, limited recursive backoff performs best
            \item Seems to outperform Generalised Language Model
            \item If significant, perhaps not enough for extrinsic evaluation
        \end{itemize}
    \end{block}
\end{frame}
\note[itemize]{
}

\begin{frame}{Results: Domains and Patterns}

    \begin{block}{Observations}
        \begin{description}
            \item[domains] Within domain evaluation yields best performance
            \item[patterns] Adding skipgrams increases performance on cross domain evaluation
        \end{description}
    \end{block}
    \vspace{-1em}
    {\small
        \begin{columns}[T,totalwidth=\textwidth]
            \begin{column}{0.5\textwidth}
                \begin{block}{Training with only $n$-grams}
                    %\rowcolors{1}{ruhuisstijlrood!12}{ruhuisstijlrood!25}
                    \begin{tabular}{rrrr}
                            & jrc & 1bw  & emea \\ \hline
                        jrc & 13  & 1195 & 961 \\
                        1bw & 768 & 158  & 945 \\
                        emea& 600 & 1143 & 4
                    \end{tabular}
                \end{block}
            \end{column}
            \begin{column}{0.5\textwidth}
                \begin{block}{and with skipgrams}
                    %\rowcolors{1}{ruhuisstijlrood!12}{ruhuisstijlrood!25}
                    \begin{tabular}{rrrr}
                            & jrc & 1bw  & emea \\ \hline
                        jrc & 13  & 1162 & 939 \\
                        1bw & 751 & 162  & 921 \\
                        emea& 581 & 1155 & 4
                    \end{tabular}
                \end{block}
            \end{column}
        \end{columns}
        \hspace{1em}
        \begin{block}{Relative differences}
            %\rowcolors{1}{ruhuisstijlrood!12}{ruhuisstijlrood!25}
            \begin{tabular}{rrrr}
                    & jrc & 1bw & emea \\ \hline
                jrc & 2.0 & \cellcolor{green!25}{-2.8} & \cellcolor{green!25}{-2.3} \\
                1bw & \cellcolor{green!25}{-2.2} & 2.4 & \cellcolor{green!25}{-2.6} \\
                emea& \cellcolor{green!25}{-3.2} & 1.1 & 0.7
            \end{tabular}
        \end{block}
    }
\end{frame}
\note[itemize]{
}

\begin{frame}{Results: Effect of Different Backoff Methods}
    \begin{block}{Observations}
        \begin{description}
            \item[backoff] For generic corpora, limited recursive backoff performs best
        \end{description}
    \end{block}
    \vspace{-1em}
    {\small
        \begin{columns}[T,totalwidth=\textwidth]
            \begin{column}{0.5\textwidth}
                \begin{block}{\hspace{2.55cm}$n$-grams\vphantom{Skipgrams}}
                    %\rowcolors{1}{ruhuisstijlrood!12}{ruhuisstijlrood!25}
                    \begin{tabular}{rrrrr}
                        & & jrc & 1bw & emea \\ \cline{3-5}
                        \multirow{3}{*}{jrc} & ngram & \cellcolor{green!25}{13} & 1510 & 1081 \\
                        & limited& 14 & 1477 & 1122 \\
                        & full & 69 & \cellcolor{green!25}{1195} & \cellcolor{green!25}{961} \\
                        &&&& \\
                        \multirow{3}{*}{1bws} & ngram & \cellcolor{green!25}{768} & \cellcolor{green!25}{158} & \cellcolor{green!25}{946} \\
                        & limited& 815 & 185 & 1025 \\
                        & full & 801 & 264 & 1039 \\
                        &&&& \\
                        \multirow{3}{*}{emea} & ngram & 769 & 1552 & \cellcolor{green!25}{4} \\
                        & limited& 779 & 1385 & \cellcolor{green!25}{4} \\
                        & full & \cellcolor{green!25}{600} & \cellcolor{green!25}{1143} & 32
                    \end{tabular}
                \end{block}
            \end{column}
            \begin{column}{0.5\textwidth}
                \begin{block}{Skipgrams}
                    %\rowcolors{1}{ruhuisstijlrood!12}{ruhuisstijlrood!25}
                    \begin{tabular}{rrr}
                        jrc & 1bw & emea \\ \hline
                        \cellcolor{green!25}{13} & 1843 & 1295 \\
                        \cellcolor{green!25}{13} & 1542 & 1149 \\
                        65 & \cellcolor{green!25}{1195} & \cellcolor{green!25}{939} \\
                        && \\
                        879 & 163 & 1105 \\
                        \cellcolor{green!25}{751} & \cellcolor{green!25}{162} & \cellcolor{green!25}{921} \\
                        768 & 252 & 988 \\
                        && \\
                        969 & 2089 & \cellcolor{green!25}{4} \\
                        838 & 1655 & \cellcolor{green!25}{4} \\
                        \cellcolor{green!25}{581} & \cellcolor{green!25}{1155} & 32
                    \end{tabular}
                \end{block}
            \end{column}
        \end{columns}
    }
\end{frame}
\note[itemize]{
}

\begin{frame}{Future Work}
    \begin{block}{Experiments}
        \begin{itemize}
            \item Validate significance by testing on multiple languages
            \item Investigate influence skipgrams with qualitative analysis
            \item When we find a more substantial drop in perplexity:
                \begin{itemize}
                    \item Machine translation experiments
                    \item Automated speech recognition experiments
                \end{itemize}
            \item Investigate multi-domain language models (DHPYPLM)
            \item Generalise skipgrams to flexgrams
            \item \ldots
        \end{itemize}
    \end{block}
\end{frame}
\note[itemize]{
}

\end{document}
