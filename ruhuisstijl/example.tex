\documentclass[t]{beamer}
\usepackage[utf8]{inputenc}
\usepackage[T1]{fontenc}
\title{RU Beamer}
\subtitle{Gebruiksvoorwaarden \& sjablonen (juli 2014)}
\date{\today}
\author{lama-fan}

%\usetheme[official=false]{ruhuisstijl}
\usetheme[official=true]{ruhuisstijl}

\begin{document}

\begin{frame}
    \titlepage
\end{frame}

\begin{frame}
    \frametitle{Gebruiksvoorwaarden algemeen}

    \begin{block}{Titelpagina's (rood)}
        \begin{itemize}
            \item Maak altijd gebruik van de rode titelpagina met het logo van de Radboud Universiteit rechtsonder
            \item Er mogen geen extra tekstvelden of afbeeldingen worden toegevoegd
            \item Indien een openingspagina als tussenpagina wordt gebruikt mogen er wel afbeeldingen toegevoegd worden
        \end{itemize}
    \end{block}

    \begin{block}{Tekstpagina's}
        \begin{itemize}
            \item Gebruik voor tekstpainga's altijd de witte dia met het logo in de rode balk
            \item Teksten linkslijnend plaatsen, niet centreren
        \end{itemize}
    \end{block}
\end{frame} 

\begin{frame}
    \frametitle{Gebruiksvoorwaarden tekstpagina's (witte achtergrond)}

    \begin{block}{Maak altijd gebruik van lettertype Arial}
        \begin{itemize}
            \item Paginatitel: standaard, grootte 30 pt
            \item Tekst/inhoud: standaard, grootte 25 pt (of 21 pt)
            \item Tussenkoppen: vet, grootte 25 pt (of 21 pt)
            \item Fotobijschriften: standaard, grootte 18 pt
        \end{itemize}
    \end{block}

    \begin{block}{Plaats teksten altijd in zwart of rood}
        \begin{itemize}
            \item Paginatitel: RU huisstijl rood \emph{(RGB: 190, 49, 26)}
            \item Inhoud tekst en tussenkoppen: zwart
        \end{itemize}
    \end{block}

    
\end{frame}

\begin{frame}
    \frametitle{Titel en object}

    \begin{block}{Dit is een tussenkop}
    Rectiatem sunto bla velesti berestrupta conseria quam quae commo et eaquam quo dolent omnistis estion cuptatet duciendae dolorunt ipit, omnimus trumqui ommolor simporuntium fugit eicatem quis autem eatemquiam nissum eatum facerit inciis voluptas quae aut et es dellab ipsum, ium alis aboriandunt ea sinverios sequo ea consedi psapid.

    Een opsomming:
    \begin{itemize}
        \item que volore non etur aut laborum, te repudam, sus es acerrov itatest omnitatur, ea vid qui tempore re, alique.
        \begin{itemize}
            \item Am restibusam nihillor 
            \item Alias ne officati officate
            \item Sequae dollitate porat vitatem
        \end{itemize}
    \end{itemize}
    \end{block}
\end{frame}

\begin{frame}
    \frametitle{Twee objecten}

    \begin{columns}[T]
        \begin{column}{0.45\textwidth}
        \begin{block}{Dit sjabloon kunt u gebruiken voor twee tekstkolommen}
            Berestrupta conseria quam quae commo et eaquam quo dolent omnistis estion cuptatet duciendae ommolor simporuntium fugit eicatem quis autem eatemquiam nissum eatum facerit inciis voluptas quae 
aut et es dellab ipsum, ium alis aboriandunt ea sinverios sequo ea consedi psapid molore autestium dio el in pelibea rcimustio esectae moluption reriam.
        \end{block}
        \end{column}
        \begin{column}{0.45\textwidth}
Am restibusam nihillor alias ne officati officate numet, quiate autem rerro ipsam, sequae dollitate porat vitatem litatiaestis acesequid et ut moluptas dolorum voluptat a poruntibus imillaut fugia velitatempor.

Magniscil illuptibus moleceria cumquis doluptu saerro in coresto volorecesse modit qui omnima volluptur, quo magnia coratis dus et faccae non plibusant. Ugit voluptatio eseria possimaio opturitatur.
        \end{column}
    \end{columns}
\end{frame}

\end{document}
